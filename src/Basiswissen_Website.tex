\documentclass[10pt,a4paper,oneside]{beamer}
\usepackage[utf8]{inputenc}
\usepackage[ngerman]{babel}
\usepackage{amsmath}
\usepackage{amsfonts}
\usepackage{amssymb}
\usepackage{graphicx}
\usepackage{hyperref}
\usepackage{listings}
\usepackage{enumitem}
\author{Sabine Loch\and Matthias Jakobi}
\title{Basiswissen Website (Plone)}
\usepackage{tikz}
\usepackage{environ}
\usepackage{sidecap}
\usepackage{wrapfig}

\makeatletter
\newcommand{\LogoStuRaHTW}{
    \input{../res/Logo_normal.pgf}
}

\newlist{tip}{enumerate}{1}
\setlist[tip]{label=Tip: ,leftmargin=*}

\newlist{hinweis}{enumerate}{1}
\setlist[hinweis]{label=Hinweis: ,leftmargin=*}

\setitemize{label=\usebeamerfont*{itemize item}%
      \usebeamercolor[fg]{itemize item}
        \usebeamertemplate{itemize item}}
\lstset{
    frame=tb, % draw a frame at the top and bottom of the code block
    tabsize=2, % tab space width
    showstringspaces=false, % don't mark spaces in strings
    numbers=left, % display line numbers on the left
    commentstyle=\color{green}, % comment color
    keywordstyle=\color{blue}, % keyword color
    stringstyle=\color{red} % string color
}
\hypersetup{
    colorlinks=true,
    linkcolor=blue,
    urlcolor=red,
    linktoc=all
}

\setbeamertemplate{headline}
{
    \leavevmode%
    \hbox{%
        \begin{beamercolorbox}[wd=\paperwidth,ht=1.0ex,dp=1ex]{secsubsec}%
            \vspace*{-4em}
            \raggedleft
            \hspace*{2em}%
            {
                    \scalebox{.08}{
                        \LogoStuRaHTW{}
                    }
            }
            %
            \hspace*{2em}%
        \end{beamercolorbox}%
    }%
}

\begin{document}
\maketitle
\frame{
    \frametitle{Quellen}
    \begin{itemize}
        \item Plone Nutzerhandbuch, Viet Schiele, \url{http://www.plone-nutzerhandbuch.de/}, Lizenz: Creative-Commons-Lizenz Namensnennung-Nicht-kommerziell-Weitergabe unter gleichen Bedingungen 2.0 Deutschland \url{http://creativecommons.org/licenses/by-nc-sa/2.0/de/}
    \end{itemize}
}
\frame{\frametitle{Inhaltsverzeichnis}\tableofcontents}

\section{Anmelden}
\frame{
    \frametitle{Anmeldung}
    \begin{figure}[!h]
        \centering
        \includegraphics[width=0.66\textwidth]{../res/plone4-anmeldeportlet.png}
    \end{figure}
    \begin{description}
        \item[Benutzername] VornameNachname
        \item[Passwort] Initiales Passwort wird per E-Mail zugesendet
    \end{description}
}

\subsection{Passwort vergessen}
\frame{
    \frametitle{Passwort vergessen}
    \begin{figure}[!h]
        \centering
        \includegraphics[width=\textwidth]{../res/plone4-passwort-zusenden.png}
    \end{figure}
    \begin{description}
        \item[Mein Benutzername ist] VornameNachname
    \end{description}
}

\section{Website Aufbau}
\frame{
    \frametitle{Website Aufbau}
    \begin{itemize}
        \item Startseite
        \item Studentinnen- \& Studentenrat
        \item weitere studentische Gremien
        \item Studentische Vertretung (an unserer Hochschule)
        \item Mitmachen
        \item Intern
        \item Hochschule
        \item Refugees Welcome
    \end{itemize}
}

\subsection{Startseite}
\frame{
    \frametitle{Website Aufbau - Startseite}
    \begin{figure}[!h]
        \centering
        \includegraphics[width=0.9\textwidth]{../res/stura-startseite.png}
        \vspace{-20pt}
    \end{figure}
    \begin{itemize}
        \item aktuelle Inhalte
        \begin{itemize}
            \item Nachrichten
            \item Termine
            \item Wikieinträge
        \end{itemize}
    \end{itemize}
}

\subsection{Studentinnen- \& Studentenrat}
\frame{
    \frametitle{Website Aufbau - Studentinnen- \& \newline Studentenrat}
    \begin{itemize}
        \item Übersicht, Struktur des StuRa
        \begin{itemize}
            \item Referate $\rightarrow$ Bereiche
            \item Sprecher*innen
            \item Referatskollegium
        \end{itemize}
    \end{itemize}
    \begin{itemize}
        \item allgemeines
        \begin{itemize}
            \item Sitzungen $\rightarrow$ Anträge $\rightarrow$ Protokolle
            \item Ordnungen
            \item Mitglieder
            \item Nachrichten
            \item Termine
        \end{itemize}
    \end{itemize}
}

\subsection{weitere studentische Gremien}
\frame{
    \frametitle{Website Aufbau - weitere studentische\newline Gremien}
    Ebenen: intern, lokal, regional, bundesweit, international
    \begin{itemize}
        \item Fachschaften
        \item Ausschüsse
        \item studentischer Hochschulrat
        \item Konferenz Sächsischer Studierendenschaften (KSS)
        \item freier zusammenschluss von studentinnenschaften (fzs)
        \item Studentischer Akkreditierungspool
    \end{itemize}
}

\subsection{Studentische Vertretungen}
\frame{
    \frametitle{Website Aufbau - Studentische Vertretungen}
    Ebenen: intern, lokal, regional, bundesweit, international
    \begin{itemize}
        \item Fakultätsräte
        \item Senat
        \item Erweiterter Senat
        \item Wahlausschuss HTW Dresden
        \item Verwaltungsrat Studentenwerk
        \item Unterstüzung
    \end{itemize}
}

\subsection{Mitmachen, Intern, Hochschule, Refugees Welcome}
\frame{
    \frametitle{Website Aufbau - Mitmachen, Intern,\newline Hochschule, Refugees Welcome}
    \begin{description}
        \item[Mitmachen] Übersicht der Stellausschreibungen
        \item[Intern] nur für angemeldete Nutzer*innen sichtbar;
            Daten, die nur StuRa-Mitgliedern zugänglich sein dürfen
        \item[Hochschule] Schnellverweis auf die Webseite der HTW Dresden
        \item[Refugees Welcome] Schnellverweis auf ein Antidis-Projekt
    \end{description}
}


\section{Suche}
\frame{
    \frametitle{Suche}
    \begin{description}
        \item[erneute Suche] Im Suchformular können Sie gegebenenfalls Ihre Suche ändern.
        \item[RSS-Feeds] Alternativ zu den Suchergebnissen können Sie auch Feeds abonnieren.
	Damit können Sie sich schnell über Änderungen in diesen Suchergebnissen informieren
    lassen.
	\item[Trefferliste]
    \end{description}
}

\subsection{Trefferliste}
\frame{
    \frametitle{Suche - Trefferliste}
    Die Suchergebnisse lassen sich noch weiter einschränken:
    \begin{itemize}
        \item nach Artikeltyp
        \item nach Datum
    \end{itemize}
    \begin{figure}[!h]
        \centering
        \includegraphics[width=\textwidth]{../res/suchformular.png}
    \end{figure}
}

\section{Inhalte hinzufügen}
\frame{
    \frametitle{Inhalte hinzufügen}
	Artikel sind dezentral im Zuständigkeitsbereich abzulegen, d.h. im passenden Ordner des Bereiches.
	Generell ist die zentrale Ablage von z.B. Nachrichten unter ``news'' zu vermeiden. 
    Artikeltypen:
    \begin{itemize}
        \item Ordner
        \item Bild
        \item Seite
        \item Datei
        \item Link
        \item Termin
        \item Nachricht
        \item Kollektion
    \end{itemize}
}
\frame{
    \frametitle{Inhalte hinzufügen}
    \begin{figure}[!h]
        \centering
    %\begin{wrapfigure}{r}{0.4\textwidth}
        %\vspace{-20pt}
        %\begin{center}
            \includegraphics[width=0.33\textwidth]{../res/artikel-hinzufuegen.png}
        %\vspace{-20pt}
        %\end{center}
        \caption{Artikeltyp hinzuf\"ugen}
    %\end{wrapfigure}
    \end{figure}
}


\subsection{Kategorisierung}
\frame{
    \frametitle{Kategorisierung}
   \begin{itemize}
	\item Nicht persönliche Artikel jeden Typs sollten so umfangreich wie möglich kategorisiert werden. 
	\item Durch die Kategorisierung tauchen die Artikel in entsprechen Kollektionen auf
   \end{itemize}
   \begin{hinweis}
   \item Mehrere Stichworte müssen mit gedrückter \textit{Strg}-Taste ausgewählt werden.
   \end{hinweis}
    \begin{figure}[!h]
        \centering
        \includegraphics[width=0.66\textwidth]{../res/stura-kategorien.png}
    \end{figure}
}


\subsection{Artikeltypen}
\frame{
    \frametitle{Artikeltyp - Ordner}
    \begin{itemize}
        \item Angaben: Name, Titel, Beschreibung
        \item Startartikel: index\_html, index.html, index.htm, FrontPage
        \item Ansicht ändern
    \end{itemize}
    \begin{figure}[!h]
        \centering
        \includegraphics[width=0.75\textwidth]{../res/ordner-hinzufuegen.png}
    \end{figure}
}

\frame{
    \frametitle{Artikeltyp - Bild}
    Anschließend öffnet sich das folgende Formular:
    \begin{figure}[!h]
        \centering
        \includegraphics[width=\textwidth]{../res/bild-hinzufuegen_2.png}
    \end{figure}
}
\frame{
    \frametitle{Artikeltyp - Bild}
    \begin{description}
        \item[Titel] Aus dem Titel wird der Kurzname oder ID des Artikels gebildet.
 Wird kein Titel angegeben, behält das Bild üblicherweise seine ursprüngliche
 ID bei.

        \item[Beschreibung] Diese wird unter anderem bei der Anzeige von Suchergebnissen verwendet.

        \item[Bild] Klicken Sie auf *Datei auswählen* um auf Ihrem lokalen Computer eine Bilddatei zum Hochladen auszuwählen.
     Sie sollten die Bilder vor dem Auswählen für die Verwendung im Web vorbereiten.
     Eine kurze Anleitung hierzu finden Sie in der Präsentation *Erweiteres Wissen Website*
     unter dem Punkt``Bilder optimieren'' .

     Nach dem Hochladen wird Ihnen dann eine Vorschau des Bildes angezeigt.
        \item[Stichworte] Da Bilder nicht textuell erschlossen werden können, kommt den Stichworten eine besondere Bedeutung zu.
    \end{description}
}

\frame{
    \frametitle{Artikeltyp - Bild}
 \begin{tip}
     \item Eine Einführung zur Verschlagwortung von Bildern finden Sie unter
 `Verschlagwortungsstrategien
 \url{https://www.veit-schiele.de/profil/artikel/verschlagwortungsstrategien}`
 \end{tip}
}

\frame{
    \frametitle{Artikeltyp - Seite}
    \begin{itemize}
        \item Angaben: Name, Titel, Beschreibung, Eigenschaften
        \item Inhalte via Webbrowser eingebar
    \end{itemize}
    Dabei kann der Text in folgenden Formaten eingegeben werden:
    \begin{description}
        \item[HTML] ermöglicht Ihnen die direkte Eingabe von HTML;
        \item[Einfacher Text] ermöglicht Ihnen die einfache Eingabe von Text.
        \item[Markdown] ermöglicht Ihnen die direkte Eingabe mit Markdown-Syntax
    \end{description}

    Optional können Sie auch Textdateien auf den Server hochladen.
}

\frame{
    \frametitle{Artikeltyp - Datei}

	Dateiobjekte können Dateien enthalten, die heruntergeladen werden können.
	Hinzufügen:
	\begin{itemize}
	\item ``Durchsuchen\dots''-Taste klicken
	\item Verzeichnis durchsuchen, Datei auswählen
	\item ``Hochladen\dots''-Taste klicken
	\end{itemize}
	Die Bezeichnung, Beschreibung und Kategorisierung der Datei sind sehr relevant, da nur durch sie das einfache Auffinden der Datei gewährleistet werden kann.
    \begin{figure}
    %\begin{wrapfigure}{o}{\textwidth}
        %\vspace{-20pt}
        %\begin{center}
            \includegraphics[width=0.5\textwidth]{../res/datei-hinzufuegen.png}
        %\vspace{-20pt}
        %\end{center}
        \caption{Datei hinzuf\"ugen}
    %\end{wrapfigure}
    \end{figure}
}

\frame{
    \frametitle{Artikeltyp - Link}
Links können als Schnellverweise gesetzt werden. 
So führt z.B. der Link \url{www.stura.htw-dresden.de/ese} für Externe direkt auf den Ordner der ESE. 
    Bitte beachten Sie bei der Angabe von externen Links die Angabe des Protokolls (z.B. ``http://'' für Webpages oder ``ftp://'' für Dateien).
}

\frame{
    \frametitle{Artikeltyp - Termin}
    Termine sind Objekte, die im Kalender eingetragen werden. Termine sollten im Ordner \textit{Termine} der ``zuständigsten'' Stelle, z.B Referat, Bereich oder Projekt eingetragen werden. Grundsätzlich sollten Termine von der Navigation ausgeschlossen werden. Dazu ist beim Artikel über Bearbeiten und Einstellungen ein Haken bei ``Von Navigation ausschließen'' zu setzen.

    \begin{itemize}
        \item Angabe: Name, Titel, Beschreibung
        \item weiter Ereignistypen, die auch als Schlagwörter verwendet werden
    \end{itemize}
    \begin{hinweis}
        \item Wollen Sie die Einträge in dieser Liste geändert haben, wenden Sie sich bitte an die Administration der Website.
    \end{hinweis}
}

\frame{
    \frametitle{Artikeltyp - Termin}
    \begin{description}
        \item[Terminanfang] Datum und Uhrzeit des Beginns; Sie können alternativ auch im nebenstehenden Kalender den Beginn auswählen.
        \item[Terminende] Datum und Uhrzeit können alternativ auch in nebenstehendem Kalender angegeben werden.
        \item[Terminort] Hier können Sie den Ort des Termins angeben.
        \item[Terminankündigung] Ankündigungstext für den Termin; alternativ können Sie auch eine Textdatei hochladen.
        \item[Teilnehmer] Teilnehmende des Termins.
        \item[Terminart] Die Art des Termins.
        \item[URL] Hier können Sie eine Web-Adresse angeben, die weitergehende Informationen über dieses Ereignis liefert.
        \item[Kontaktname] Kontaktperson oder -organisation für das Ereignis.
        \item[Kontaktadresse] Adresse, bei der Sie weitergehende Informationen zum Ereignis erhalten können.
        \item[Kontakt-E-Mail] E-Mail-Adresse für Nachfragen.
        \item[Kontakt-Telefon] Telefonnummer für Nachfragen und Reservierungen.
    \end{description}
}

\frame{
    \frametitle{Artikeltyp - Nachricht}
    Nachrichten sind Seiten  mit Titel, optionaler Beschreibung und Bild.

    Für Nachrichten lassen sich im Gegensatz zu Seiten noch je ein Bild mit Bildtitel angeben.
    \begin{figure}[!h]
        \centering
        \includegraphics[width=0.75\textwidth]{../res/nachricht-hinzufuegen.png}
    \end{figure}
}

\frame{
    \frametitle{Artikeltyp - Kollektion}
    Kollektionen sind vorgefertigte Suchanfragen, die auch die Erschließung großer Datenbestände erlauben.

Kollektionen können üblicherweise nicht von allen Mitgliedern eines Portals hinzugefügt werden, sondern nur von denjenigen, die die Rollen *Website-Administrator* oder *Verwalter* innehaben. siehe: \url{www.stura.htw-dresden.de/stura/ref/verwaltung}
}

\section{Veröffentlichung von Inhalten}
\frame{
 \frametitle{Veröffentlichung von Inhalten}
 Nicht jede*r Plonenutzer*in darf veröffentlichen 
 Für diese User übernimmt Referat ÖA die externe Veröffentlichung
 \subsection{Workflow}
 \begin{itemize}
 \item Referat xy erstellt Termin/Beitrag nach den inhaltlichen Regeln des StuRa
 \item Der Termin/Beitrag wird zur Veröffentlichung eingereicht
 \item Referat ÖA prüft Termin/Beitrag und nimmt bei Bedarf kleinere redaktionelle Änderungen und Korrekturen vor.
(Bei groben Fehlern wird der Termin/Beitrag zurückgewiesen)
\item Referat ÖA überprüft Kategorisierung und Relevanz für zusätzliche Veröffentlichung auf den Social       Media Sites.
\item Referat ÖA veröffentlicht den Termin/Beitrag extern.
\item Referat xy ist für die Verfolgung seiner Termine/Beiträge eigenverantwortlich.
\end{itemize}
}
\frame{
	\frametitle{Veröffentlichung von Inhalten}
    \begin{figure}[!h]
        \centering
        \includegraphics[height=0.8\textheight]{../res/veroeffentlichung.png}
    \end{figure}
}


\subsection{Inhalt}
\frame{
 \frametitle{Inhalt}
 Was muss vermieden werden:
 \begin{itemize}
 \item Generisches Maskulinum
 \item Fehler jeder Art
 \item Falsche Daten oder Ortsbezeichnungen
 \item Bilder ohne notwendige Rechte
 \item Falsche Verwendung von Logos
 \item Falsche oder fehlende Kategorisierung
 \item Satzzeichen, die in Rudeln auftreten 
 \end{itemize}
}

\section{Mein Ordner einrichten}
\frame{
	\frametitle{Mein Ordner einrichten}
	\begin{itemize}
		\item Mein Menue $\rightarrow$ Mein Ordner
		\item Seite hinzufügen
		\item Inhalte erstellen:
		\begin{itemize}
			\item strukturiert (mit Überschriften)
			\item Vorstellung
			\item Bild (muss kein Selfie sein)
			\item e-Mail Adresse im StuRa (mit Hyperlink)
			\item sonstiges
		\end{itemize}
		\item zur Veröffentlichung einreichen
	\end{itemize}
    Beispiele wie deine Seite aussehen kann findest du unter
    \url{https://www.stura.htw-dresden.de/members}.
}


\end{document}
